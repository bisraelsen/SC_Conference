\section{Introduction}
Autonomy defines the ability of a physical system to perform a complex tasks without human intervention for extended periods of time. This requires at least one or more of the capabilities of an artificially intelligent physical agent: reasoning, knowledge representation, planning, learning, perception, control, and communication \cite{Israelsen2017-ym}. 
Yet, an autonomous physical system (APS) always interacts with a human user in some way \cite{Bradshaw2013-ck}.  That is, the aforementioned capabilities are only the means by which an APS achieves some intended degree of self-sufficiency and self-directedness for tasks \emph{delegated} by a user to meet an `intent frame' (desired set of goals, plans, constraints, stipulations, and/or value statements) \cite{Miller2014-av}. 
As capabilities have continued to improve and find new applications, APS are also becoming more complex, opaque and difficult for users (as well as designers and other stakeholders) to fully comprehend. For APS characterized by uncertainty-based AI and data-driven machine learning, it can become extremely difficult to reconcile predictions about behavior and performance limits with desired intent frames in noisy, untested, and `out of scope' task conditions. This ushers in questions related to user trust in autonomous systems, i.e. a user's willingness in depending on an APS to carry out a delegated set of tasks, in light of the user's own (often limited and non-expert) understanding of the APS capabilities. 

This work examines how APS can be designed to actively adjust users' expectations and understanding of APS capabilities which inform trust. As surveyed in \cite{Israelsen2017-ym}, several broad classes of \emph{algorithmic assurances} for APS have been developed for this purpose. 
Assurances are challenging to develop because they must allow users to gain better insight and understanding of APS behaviors for effectively managing operations, without undermining autonomous operations or burdening users. 
Process-based assessment and meta-analysis techniques allow APS to self-qualify their capabilities and competency boundaries, by evaluating and reporting their associated degree of `self-trust' or \emph{self-confidence} for a particular task. 
%This resonates with the concept put forth by \cite{Hutchins2015-if}, which proposed using human expert evaluations of specific APS capabilities to manually encode where and when these may break down in particular tasking situations. However, to be useful in real applications, APS must be able to form these evaluations on their own. 
Several definitions and algorithmic approaches have been proposed recently which enable APS to automatically generate self-confidence scores in the context of different tasks and uncertainty-based capability assessments \cite{Sweet2016-tz, Israelsen2017-ym}. This work restricts attention to the following definition of self-confidence, which captures the others to a large extent: \textit{an agent's perceived ability to achieve assigned goals (within a defined region of autonomous behavior) after accounting for (1) uncertainties in its knowledge of the world, (2) uncertainties of its own state, and (3) uncertainties about its reasoning process and execution abilities.} However, a framework for computing and communicating self-confidence for general decision-making autonomy architectures has yet to be established. Furthermore, it has not yet been confirmed experimentally whether (or in what contexts) human-APS interfaces that incorporate self-confidence reporting improve the ability of users to delegate of tasks within APS competency limits, compared to status quo interfaces that do not use such reporting. This paper provides results aimed at addressing both of these gaps. 

%\nisar{can probably trim this parag down a bit, after editing per comment at end...}

In Section 2, we motivate the machine self-confidence idea and ground the concept for autonomous planning problems in the context of general Markov Decision Processes (MDPs), using a concrete uncertain navigation application example where tasks are delegated by a user. We review a factorization based framework for self-confidence assessment (\famsec), and in Section 3 present a novel strategy for computing the so-called `solver quality' factor of self-confidence, drawing inspiration from empirical hardness modeling literature \cite{Leyton-Brown2009-yr}. Numerical examples for the UGV navigation problem under different MDP solver, parameter, and environment conditions, indicate that the self-confidence scores exhibit desired properties. Section 4 describes the setup and initial outcomes of experimental trials to investigate the effects of reporting self-confidence to users on simulated instances of the UGV navigation problem. The results show significantly improved delegated task performance outcomes in conditions where self-confidence feedback is provided to users vs. conditions where no self-confidence feedback is provided, providing favorable evidence for the efficacy of self-confidence reporting. Section 5 presents conclusions and avenues for future investigation. %\nisar{only mention solver quality -- what about outcome assessment? this is also in the data...}


%%%---------------------------------OLD INTRO: 
% \nisar{note to self: to reduce repetetive motivation, could try to merge parts of this section with initial bit of background/related work...so that Famsec stuff goes to section 2 and save some space?}
% Thanks to advances in AI and machine learning, unmanned autonomous physical systems (APS) are poised to tackle complex decision making problems for high-consequence applications, such as wilderness search and rescue, transportation, agriculture, remote science, and space exploration. APS are ideally designed to be self-sufficient and to make self-guided decisions about complex problems delegated by users. Hence, APS 
% %that are taskable---able to translate high-level commands into suitable processes for sensing, learning, reasoning, communicating, and acting---
% must also be cognizant and knowledge-rich, i.e. capable of reasoning about the capabilities and limitations of their own reasoning and decision-making processes, anticipating possible failures, and able to recognize when they are operating incorrectly \cite{David2016-nu}.

% This work is motivated by the need to develop new computational strategies for assessing when an APS reaches its \emph{competency boundaries}. If computed and communicated correctly, such assessments can provide users with clearer predictions of APS behavior and understanding of actual APS capabilities. This can not only allow APS to take initiatives to stay within its competency boundary in untested situations, but also provide users/stakeholders with assurances that allow them to properly calibrate trust in (and hence make proper use of) intelligent APS \cite{Israelsen2017-ym}. These properties are especially important for APS that must rely heavily on non-deterministic algorithms for decision-making under uncertainty, i.e. to efficiently make approximate inferences with imperfect models, learn from limited data, and execute potentially risky actions with limited information, especially when computational models and approximations are expected to break down in unforeseen ways. %Most recent work on algorithmic introspection and meta-reasoning to date has focused on outcome-based analyses for  AI/learning agents with narrow well-defined tasks. Yet, holistic process-based techniques for algorithmic competency boundary self-assessment are also needed to accommodate broader classes of APS operating in complex, dynamic and uncertain real-world settings---whose computational models and approximations are expected to break down in less obvious or foreseen ways. 
% %%\brett{this is probably the end of the `general' introduction, I think most of the material below should be removed as it is specific to the FAT* paper \cite{Israelsen2018-qz}}


% %\nisar{for me todo: trim and merge this bit of introduction}
% This paper presents and builds on a recently developed algorithmic framework for computing and evaluating self-assessments in APS that leads to shorthand metrics of \emph{machine self-confidence}. Self-confidence is defined as an APS' perceived ability to achieve assigned goals after accounting for uncertainties in its knowledge of the world, its own state, and its own reasoning and execution abilities \cite{Aitken2016-cv,Aitken2016-fb,Sweet2016-tz}. Algorithmic computation of self-confidence is strongly linked to model-based assessments of probabilities pertaining to task outcomes and completion---but crucially goes further to provide insight into how well an APS's processes for decision-making, learning, perception, etc. are matched to intended tasks \cite{Hutchins2015-if}. 
% We argue that the short-hand insight provided by self-confidence assessments can serve as a feedback signal to anticipate degraded, nominal, or enhanced APS performance, %adapt autonomous behavior, 
% and thereby can be used to adjust user trust in APS for uncertain task settings. 
% %


%-------------------------------------OLD TEXT (IGNORE)
% \nisar{for me todo: reframe/polish statement of contributions...this is too technical/ in the weeds as written, e.g. gets into MDPs which are not yet defined, and mentions EHMs, but no one will get that...}
% The main contributions of this paper include: 1) A formal definition of `solver-quality' which is one of several factors that make up self-confidence. Herein, solver-quality is presented as a metric for assessing the suitability of approximate Markov Decision Process (MDP) solvers for a given task. 2) Solver-quality is then derived borrowing inspiration from empirical hardness models (EHMs \cite{Leyton-Brown2009-yr}. 3) Solver-quality is then evaluated using numerical experiments. The paper is organized as follows: In Section~\ref{sec:background} we further explore motivations and background for self-confidence, including concepts like trust between humans and autonomous systems, and a useful example application. \nisar{what are the results/main takeaways?? what's the payoff if people keep reading the paper?}
%
% This paper is organized as follows. In Section~\ref{sec:self-confidence} Factorized Machine Self-Confidence (\famsec) is introduced and a framework outlined. At the end of Section~\ref{sec:self-confidence} we turn our attention to one of the \famsec{} factors, `Solver Quality', and outline specific challenges and desiderata in the context of the broadly useful family of Markov Decision Process (MDP)-based planners \nisar{defined after used above!}. A learning-based technique for computing solver quality factors in MDP-family planners is then derived in Section~\ref{sec:methodology}. In Section~\ref{sec:results} we present results from numerical experiments for an unmanned autonomous vehicle navigation problem. Finally we present conclusions in Section~\ref{sec:conclusions}.

%%removed background/related work and put into different .tex file
