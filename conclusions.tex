\section{Conclusions} \label{sec:conclusions}
Unmanned autonomous physical systems are able to tackle complex decision making problems for high-consequence applications, but in order to be able to reduce the amount of supervision required these systems need to be able to perform self-assessment, or introspection. We draw on \emph{Factorized Machine Self-Confidence (\famsec)} which is a framework of self-assessments that enable an APS to quantify its own capabilities.

Herein, we have motivated and introduced two \famsec{} factors: `Outcome Assessment' (\xP{}, Sec.~\ref{sec:OA}), and `Solver Quality' (\xQ{}, Sec.~\ref{sec:SQ}), that are meant to influence a human user's trust and behavior. Preliminary findings from an Amazon Mechanical Turk experiment indicate that the presence of \xQ{} and/or \xP{} significantly improves a user's ability to perform as a dispatch supervisor for the Donut Delivery task.

Future work involves further analyzing the experimental results to identify effects of \xQ{} and \xP{} on user behavior and attitudes. Also, In condition 4 of the experiment \xQ{} and \xP{} were presented separately; it would be interesting to investigate how those might be combined into a single summary metric. Something of that nature would become even more critical as other \famsec{} metrics are developed since it is difficult for users to process many different metrics simultaneously. Another obvious direction for future work is to develop approaches for the remaining three \famsec{} factors. Each of the individual factors reflects a critical meta-assessment of the competency of the APS.
