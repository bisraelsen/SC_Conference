\section{Conclusions} \label{sec:conclusions}
Autonomous physical systems (APS) are able to tackle complex decision making problems for high-consequence applications. Still, humans always have some interaction with an APS at some level, and it can be extremely difficult for users reconcile predictions about behavior and performance limits in noisy, untested, and `out of scope' task conditions.

Herein, we investigate how to use two Factorized Machine Self-Confidence (\famsec) factors---`solver quality' (\xQ{}) and `outcome assessment' (\xO)---as \emph{algorithmic assurances} to aid users of autonomous systems. This is accomplished because \xQ{} and \xO{} reflect different competencies of an autonomous decision-making agent, thus enabling users to make more appropriate decisions regarding their use.

Preliminary findings from an Amazon Mechanical Turk experiment indicate that the presence of \xQ{} and/or \xP{} significantly improves a user's ability to perform as a dispatch supervisor for the Donut Delivery task. This is reflected by an increased total score on the experimental task when \xQ{} and/or \xO{} are presented as opposed to when they are not.

Future work involves further analyzing the experimental results to identify effects of \xQ{} and \xP{} on user behavior and attitudes. Also, In condition 4 of the experiment \xQ{} and \xP{} were presented separately; it would be interesting to investigate how those might be combined into a single summary metric. Something of that nature would become even more critical as other \famsec{} metrics are developed since it is difficult for users to process many different metrics simultaneously. Another obvious direction for future work is to develop approaches for the remaining three \famsec{} factors. Each of the factors reflects a critical meta-assessment of the competency of the APS. 
