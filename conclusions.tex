\section{Conclusions} \label{sec:conclusions}
%Autonomous physical systems (APS) are able to tackle complex decision making problems for high-consequence applications. Still, humans always have some interaction with an APS at some level, and it can be extremely difficult for users reconcile predictions about behavior and performance limits in noisy, untested, and `out of scope' task conditions.
%
We have reviewed, developed and investigated how Factorized Machine Self-Confidence (\famsec{}) factors can be implemented as \emph{algorithmic assurances} to aid users of autonomous systems. It is argued that, since these factors provide meta-cognitive self-assessments of different aspects of decision-making under uncertainty, reporting these factors will allow users to better task and use autonomous systems.  Preliminary findings from a human participant study indicated that the reporting of two \famsec{} factors, \xQ{} (`solver quality') and \xO{} (`outcome assessment'), significantly improved users' ability to perform as a dispatch supervisors for the Donut Delivery task. This is reflected by an increased total score on the experimental task when \xQ{} and/or \xO{} are presented vs. when they are not.

Future work involves further analyzing the experimental results to identify effects of \xQ{} and \xO{} on user behavior and attitudes. Also, in condition 4 of the experiment \xQ{} and \xO{} were presented separately; it would be interesting to investigate how those might be combined into a single summary metric. Such a combination might become even more critical for complex tasks or as other \famsec{} metrics are incorporated, since it may become difficult for users to process many different metrics simultaneously. Finally, future work will identify strategies for computing other \famsec{} factors for MDP-based autonomy, as well as explore other contexts in which \famsec{} factors can be computed, reported, and experimentally evaluated.  %. to meta-assessment of the competency of the APS. 
